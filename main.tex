% !TEX program = xelatex
\documentclass[bachelor,openright,notchinese]{sustcthesis}
% 默认twoside 双面打印
% 将master修改为bachelor, doctor or master
% 要使用adobe字体,添加adobefonts选项
% 使用euler数学字体,如不愿使用,去掉euler
% 使用外文写作,请添加notchinese

% 设置图形文件的搜索路径
\graphicspath{{figures/}}

% 用到的宏包
\usepackage{algorithm2e}
% 阻止hyperref宏包影响tableofcontent内容
\makeatletter
\let\Hy@linktoc\Hy@linktoc@none
\makeatother

%%%%%%%%%%%%%%%%%%%%%%%%%%%%%%
%% 封面部分
%%%%%%%%%%%%%%%%%%%%%%%%%%%%%%
 % 中文封面内容
  \title{毕业论文}%一般情况下扉页和封皮、书脊共用一个标题文本,可以不用定义\spinetitle(仅硕博有用), \covertitle(本硕博均有用)和\encovertitle(仅本科有用)。特殊情况见下。
  %特殊情况1:本例中\title命令里含有换行控制字符,这会导致制作书脊的时候出现错误,例如如果你注释掉\spinetitle{...}这一行就会报错。这时需要定义一个不含换行等命令的\spinetitle,这并不表示\spinetitle里不能有任何命令——只能使用有限的命令。
  %特殊情况2:本例中标题过长,所以需要缩小书脊标题的字号。
  %特殊情况3:本例中中英文混排,由于tex竖排的原理限制,中英文基线不重合,所以需要人工调整英文的基线。具体调整量根据不同字体有所不同。
  %\covertitle{六次循环域}
  %\covertitle{中文题目第一行\\中文题目第二行}
  %不要在此调整封皮字体大小! Do not set Cover Page font size here!
  %特殊情况4:本例中\title中含有多个换行,导致标题超过了两行。根据制本厂规定,封皮标题不能超过两行。因此需要定义封皮使用的标题\covertitle. 如果你注释掉这一行,就会发现封皮不符合规定。
  %\encovertitle{On Cyclic Sextic Fields}
  %\encovertitle{English Title Line 1\\English Title Line 2\\English Title Line 3}
  %不要在此调整封皮字体大小! Do not set Cover Page font size here!
  %特殊情况5:仅本科生有用。本科封皮中有英文标题,不超过三行。与上类似。
  \author{张 \ 三 }
  \depart{计算机科学与工程}%系别,硕博请用系代号,本科请用全称如
  \major{计算机专业}%专业,硕博请用全称,本科不需要
  \advisor{XXXX \ 教授}
  % \coadvisor{XXX\ 教授,\ XXX\ 教授}%第二导师,没有请注释掉
  \studentid{XXXXX}%For bachelor only
  \submitdate{二〇一九年三月}

  % 英文封面内容
  \entitle{Bachelor Thesis}
  \enauthor{Alan Mathison Turing}
  \studentid{XXXX}
  \endepart{Computer Science and Engineering}
  \enmajor{Computer Science}
  \enadvisor{Prof. XXX}
  % \encoadvisor{}
  % \encoadvisorsec{}
  \ensubmitdate{March, 2019}
  
\begin{document}

% 封面
\maketitle

%特别注意,以下述顺序为准,在对应部分添加文档部件,切勿颠倒顺序:
%本科论文的文档部件顺序是:
%    frontmatter:致谢、目录、中文摘要、英文摘要、
%    mainmatter: 正文章节
%    backmatter: 参考文献或资料注释、附录
%%%%%%%%%%%%%%%%%%%%%%%%%%%%%%
%% 前言部分
%%%%%%%%%%%%%%%%%%%%%%%%%%%%%%
\frontmatter
\makeatletter

\ifustc@bachelor
	%%%%%%%%%%%%%%%%%
	%本科论文修改这里
	%%%%%%%%%%%%%%%%%
	% 致谢
	\chapter*{诚信承诺书}
\label{chap:honest}

\begin{enumerate}
\item 本人郑重承诺所呈交的毕业设计(论文),是在导师的指导下,独立进行研究工作所取得的成果,所有数据、图片资料均真实可靠。
\item 除文中已经注明引用的内容外,本论文不包含任何其他人或集体已经发表或撰写过的作品或成果。对本论文的研究作出重要贡献的个人和集体,均已在文中以明确的方式标明。
\item 本人承诺在毕业论文(设计)选题和研究内容过程中没有抄袭他人研究成果和伪造相关数据等行为。
\item 在毕业论文(设计)中对侵犯任何方面知识产权的行为,由本人承担相应的法律责任。
\end{enumerate}

\vskip 3\baselineskip


\begin{flushright}

作者签名: \underline{\hspace{4cm}}
\vskip \baselineskip
\underline{\hspace{1.4cm}}年\underline{\hspace{0.7cm}}月\underline{\hspace{0.7cm}}日

\end{flushright}

	\chapter{Preface}
\label{chap:chap-preface}
\vskip 28pt

\begin{flushright}

Alan Mathison Turing

March, 2019 at SUSTech

\end{flushright}






     
	
	
	%目录部分
	%目录
	\tableofcontents
	%默认表格、插图、算法索引名称分别为“表格索引”、“插图索引”和“算法索引”
	%如果需要自行修改lot,lof,loa的名称,请定义
	%\ustclotname{...}
	%\ustclofname{...}
	%\ustcloaname{...}

	% 表格索引
	%\ustclot
	% 插图索引
	%\ustclof
	%算法索引 
	%如果需要使用算法环境并列出算法索引,请加入补充宏包。
	%\ustcloa
	
	% 摘要
	% \begin{cnabstract}
%     本文主要说明了....

% \keywords{关键词1, 关键词2}
% \end{cnabstract}

\begin{enabstract}

Python is a popular dynamic programming language that has thrived in the past decade with massive applications in various deciplines.

\enkeywords{Python, API Evolution, tree differencing, dynamic programming language}

\end{enabstract}
%此文件中含有中英文摘要
  \begin{denotation}
\item[$\mathbb{Q}$] rational number field
\end{denotation}

    
\else
	%%%%%%%%%%%%%%%%%
	%硕博论文修改这里
	%%%%%%%%%%%%%%%%%
	% 摘要
	% \begin{cnabstract}
%     本文主要说明了....

% \keywords{关键词1, 关键词2}
% \end{cnabstract}

\begin{enabstract}

Python is a popular dynamic programming language that has thrived in the past decade with massive applications in various deciplines.

\enkeywords{Python, API Evolution, tree differencing, dynamic programming language}

\end{enabstract}
%此文件中含有中英文摘要
	% 目录
	\tableofcontents
	%默认表格、插图、算法索引名称分别为“表格索引”、“插图索引”和“算法索引”
	%如果需要自行修改lot,lof,loa的名称,请定义
	%\ustclotname{...}
	%\ustclofname{...}
	%\ustcloaname{...}

	% 表格索引
	\ustclot
	% 插图索引
	\ustclof
	%算法索引 
	%如果需要使用算法环境并列出算法索引,请加入补充宏包。
	%\ustcloa
	
	%符号说明,需要加入补充包
	\begin{denotation}
\item[$\mathbb{Q}$] rational number field
\end{denotation}
%不是必需的,如果不想列出请注释掉
\fi
\makeatother

%%%%%%%%%%%%%%%%%%%%%%%%%%%%%%
%% 正文部分
%%%%%%%%%%%%%%%%%%%%%%%%%%%%%%
\mainmatter
  \chapter{Introduction}
\label{chap:introduction}

Python is a popular dynamic programming language. Development frameworks written in Python thrived across multiple deciplines in recent decades, including \textit{TensorFlow} for deep learning, \textit{Pandas} for data analytics, and \textit{Django} for web services. The rule of Continuing Change discloses that programs either undergo continual change or become progressively less useful over time~\cite{evo-laws}. Akin to frameworks developed in other programming languages, Python frameworks obey this rule. However, the complication of framework release versions induce compatibility issues when the invoked APIs do not align with the APIs installed. Using the wrong version of framework APIs might induce compilation or runtime problems.

The goal of this project is to provide better API usage warnings for Python programs using static analysis prior to execution. Previously when client software developers called an obsolete API, Python runtime would print traces that report unavailable attribute in the module, which might be the consequence of multiple causes. We aims to understand the changes in framework implementations at a high level close to the framework developers' original intents, which requires us to understand compound changes in addition to atomic ones, and provide client programmers with useful suggestions such as "The API you used is renamed to ..." or "The API you used is obsoleted."
To summarize, this thesis accomplishes two major tasks:

\begin{itemize}
  \item Analyzed three real-world Python frameworks and collected common types of compound changes occurred in Python API evolution.
  \item Designed and implemented a tool to automatically detect compound changes classified in the above empirical study.
\end{itemize}
	
  %自行添加
  \chapter{}
\label{chap:chap-two}
  \chapter{Conclusion}
\label{chap:conclusion}

In this thesis, we conducted an empirical study of commit histories of real-world Python frameworks to understand the high-level compound changes in Python framework API evolution. We collected 535 commits that represented typical compound change patterns in three popular Python libraries. Based on our empirical findings, we designed and implemented a tool that can automatically detect compatibility issues in client programs caused by compound breaking changes in dependent Python frameworks. \textit{And will evaluate it on real-world projects to measure its capability.}
 


%%%%%%%%%%%%%%%%%%%%%%%%%%%%%%
%% 附件部分
%%%%%%%%%%%%%%%%%%%%%%%%%%%%%%
\backmatter
  %结语
  
  % 参考文献
  % 使用 BibTeX
  % 选择参考文献的排版格式。注意ustcbib这个格式不保证完全符合要求,请自行决定是否使用
  \bibliographystyle{sustcbib}%{GBT7714-2005NLang-UTF8}
  \bibliography{bib/tex}
  \nocite{*} % for every item
  % 不使用 BibTeX
  % \include{chapter/bib}
  % 附录,没有请注释掉
  \begin{appendix}

  \chapter{Experiment Results}
\label{chap:appA}
Some descriptions here.
\section{Subsection}

  
  \end{appendix}
  \begin{thanks}

  This work is supervised and guided by Professor Yepang Liu of Southern University of Science and Technology (SUSTech), Professor Ming Wen of Huazhong University of Science and Technology (HUST), Hengcheng Zhu of Hong the Hong Kong University of Science and Technology (HKUST), and Zhaoxu Zhang of the University of South California (USC). We discussed in group meetings on a weekly basis.

\vskip 18pt

\begin{flushright}

Junda Ai

May, 2021

\end{flushright}

\end{thanks}

\end{document}
