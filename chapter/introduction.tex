\chapter{Introduction}
\label{chap:introduction}

Python is a popular dynamic programming language. Development frameworks written in Python thrived across multiple disciplines in recent decades, including \textit{TensorFlow} for deep learning, \textit{Pandas} for data analytics, and \textit{Django} for web services. The rule of Continuing Change discloses that programs either undergo continual change or become progressively less useful over time~\cite{evo-laws}. Akin to frameworks developed in other programming languages, Python frameworks obey this rule. However, the complication of framework release versions induces compatibility issues when the invoked APIs do not align with the APIs installed. Using the wrong version of framework APIs might induce compilation or runtime problems.

The goal of this project is to provide better API usage warnings for Python programs using static analysis prior to execution. Previously when client software developers called an obsolete API, Python runtime would print traces that report unavailable attributes in the module, which might be the consequence of multiple causes. My goal is to automatically detect the changes in framework implementations at a high level close to the framework developers' original intents, which requires us to understand compound changes in addition to atomic ones.
To summarize, this thesis accomplishes two major tasks:

\begin{itemize}
  \item Analyzed three real-world Python frameworks and collected common types of compound changes that occurred in Python API evolution.
  \item Designed and implemented a tool \textit{ccdetector} to automatically detect high-level changes classified in the above empirical study.
\end{itemize}
